\documentclass[a4paper,12pt,french]{article}
\usepackage[utf8]{inputenc}
\usepackage[T1]{fontenc}
\usepackage{babel}
\usepackage{amssymb}
\usepackage{mathrsfs} 
\usepackage{enumerate}
\usepackage{amsthm}
\usepackage{amsmath}
\usepackage{longtable}
\usepackage{caption}
\usepackage{relsize}
\usepackage{xcolor}
\usepackage{enumitem}
\usepackage[document]{ragged2e}
\usepackage{hyperref}
\usepackage{etoolbox}
\usepackage{geometry}
\usepackage{graphicx}
\usepackage{xspace}
\usepackage{epstopdf}
\usepackage[nottoc, notlof, notlot]{tocbibind}
\usepackage{fourier}

%opening
\title{DM RNDP}
\date{}
\author{Jules Berry, Thomas Poisson}

\theoremstyle{definition}
\newtheorem{definition}{D\'efinition}[section]
\AtEndEnvironment{definition}{\hfill $\square$}
\newtheorem{remarque}{Remarque}[section]

\theoremstyle{theorem}
\newtheorem{prop}{Proposition}[section]
\newtheorem{thm}{Th\'eor\`eme}[section]
\newtheorem{lemme}{Lemme}[section]
\newtheorem{corollaire}{Corollaire}[section]


\renewcommand\qedsymbol{$\blacksquare$}
\newcommand{\T}{\mathbb{T}}
\newcommand{\norm}[2]{\lVert{#1}\rVert_{#2}}
\newcommand{\sgn}{\text{sgn}}
\newcommand{\Lp}[1]{L^{#1}(\T)}
\newcommand{\C}{\mathcal{C}}



\begin{document}

\maketitle

\justify

\tableofcontents

\newpage

\section{Préliminaires}
\subsection{Question 1}
 \item On cherche une fonction $u \in \C^2([-1,1])$ telle que 
 \begin{equation}\label{eq:problème_1}
  \begin{cases}
   -(\alpha(x) u'(x))' + \gamma(x)u(x) = f(x) \quad \forall x \in ] -1, 1[, \\
   u(-1) = c_1,\ u(1) = c_2
  \end{cases}.
 \end{equation}

 Soit $u$ un solution de ce problème et on pose 
 \[
  v(x) = u(x) - c_1 \frac{x - 1}{2} - c_2 \frac{x+1}{2}
 \]
cette fonction est alors aussi $\C^2([-1,1])$ et on a de plus 
\[
 v'(x) = u'(x) - \frac{c_1 + c_2}{2}, \quad v''(x) = u''(x).
\]
En injectant cette fonction dans l'EDO de \eqref{eq:problème_1} on a
\begin{align*}
 -(\alpha(x)v'(x))' + \gamma(x)v(x) &= -\alpha'(x)v'(x) - \alpha(x)v''(x) + \gamma(x)v(x) \\
 &= -(\alpha(x)u'(x))' + \frac{c_1 + c_2}{2}\alpha'(x) + \gamma(x)u(x) - \gamma(x)\left (c_1 \frac{x-1}{2} + c_2 \frac{x+1}{2} \right)
\end{align*}
On a donc que la fonction $v$ est solution du problème
\begin{equation}\label{eq:problème_2}
 \begin{cases}
  -(\alpha(x)v'(x))' + \gamma(x)v(x) = \tilde f(x) \\
  v(-1) = v(1) = 0
 \end{cases}
\end{equation}
où $\tilde f(x) = f(x) + \gamma(x)\left(c_1 \frac{x-1}{2} + c_2 \frac{x+1}{2}\right) - \frac{c_1 + c_2}{2}\alpha'(x)$.

De plus il est clair que $v(-1) = v(1) = 0$ et comme $f \in L^2(]-1,1[)$ , $\alpha \in \C^1([-1,1])$ et $\gamma \in L^\infty ([-1,1]) \subset L^2(]-1,1[)$ on a aussi $\tilde f \in L^2([-1,1])$. 

On voit alors que le problème \eqref{eq:problème_2} est de la forme souhaitée. De plus si $v$ est une solution de ce problème et en posant $w(x) = v(x) + c_1 \frac{x - 1}{2} + c_2 \frac{x+1}{2}$, on peut voir que $w$ est une solution de \eqref{eq:problème_1}. On a donc montré que l'étude de ce dernier se ramène à l'étude de \eqref{eq:problème_2}.

\subsection{Question 2}

On pose $u = \rho v$ où $\rho : x \in [-1,1] \mapsto \exp \left ( \frac{1}{2} \int_{-1}^x \frac{\beta(t)}{\alpha(t)}\ dt \right)$.
Alors si $u$ est une solution de l'EDO 
\begin{equation}
 -(\alpha u')' + \beta u' + \gamma u = f
\end{equation}
On remarque alors que 
\begin{align*}
 \rho'(x) &= \frac{d}{dx} \exp \left ( \frac{1}{2} \int_{-1}^x \frac{\beta(t)}{\alpha(t)}\ dt \right) = \frac{\beta(x)}{2\alpha(x)} \exp \left ( \frac{1}{2} \int_{-1}^x \frac{\beta(t)}{\alpha(t)}\ dt \right) \\
 &= \frac{\beta(x)}{2\alpha(x)} \rho(x).
\end{align*}
Et par suite 
\[
 \rho'' = \frac{1}{2} \left ( \frac{\beta' \alpha - \beta \alpha'}{\alpha^2}\rho + \frac{\beta}{\alpha} \rho'\right) = \frac{\beta' \alpha + 2 \beta^2 - \beta \alpha'}{2 \alpha^2}\rho.
\]

En réétudiant l'EDO on trouve donc 
\begin{align*}
 -(\alpha u')' + \beta u' + \gamma u &= - \alpha' (\rho' v + \rho v') - \alpha ( \rho '' v + 2 \rho' v' + \rho v'') + \beta (\rho' v + \rho v') + \gamma \rho v \\
 &= \rho \left ( -\frac{\alpha' \beta v}{\alpha} - \alpha' v' - \frac{\beta' v}{2} - \frac{\beta^2 v}{\alpha} + \frac{\beta \alpha' v }{\alpha} - \beta v' - \alpha v'' + \frac{\beta^2 v}{\alpha} + \beta v' + \gamma v \right) = f
\end{align*}
Par définition de la fonction de $\rho$ celle-ci est strictement positive et on peut donc diviser par $\rho$ dans l'équation. Alors en simplifiant le membre de gauche on trouve
\[
 - \alpha' v' - \alpha v'' - \frac{1}{2} \beta' v +\gamma v = \frac{f}{\rho}.
\]
Alors en posant $\delta = \gamma - \frac{1}{2}\beta'$ et $g = \frac{f}{\rho}$. On trouve donc 
\begin{equation}\label{eq:EDO_1}
 -(\alpha v')' + \delta v = g.
\end{equation}
On remarque au passage que nous avons supposé que la fonction $\beta$ était au moins $\C^1$.

\subsection{Question 3}

On considère l'EDO $-(\alpha u')' + \gamma u = f$ sur $[a,b]$. En posant 
\[
 h : x \in [-1,1] \mapsto a \frac{x-1}{2} + b \frac{x+1}{2}
\]
et on considérant la fonction $v : x \in [-1,1] \mapsto u \circ h(x)$ on trouve que $v(-1) = u(a)$ et $v(1) = u(b)$. De plus en posant aussi $\tilde \alpha = \alpha \circ h$, $\tilde \gamma = \frac{b - a}{2} \gamma \circ h$ on a en supposant que $u$ est une solution de l'EDO
\begin{align*}
 -(\tilde \alpha(x) v'(x))' + \tilde \gamma(x) v(x) &= -(\alpha(h(x))(u(h(x)))')' + \frac{a + b}{2} \gamma(h(x))v(h(x)) \\
 &= -\frac{a + b}{2}( \alpha (h(x)) u'(h(x)))' + \frac{a + b}{2} \gamma(h(x)) u(h(x)) \\
 &= \frac{a+b}{2} f(h(x)).
\end{align*}
On peut alors poser $\tilde f(x) = \frac{a+b}{2} f \circ h (x)$ et on a alors que $v$ est solution de la nouvelle EDO 
\[
 -(\tilde \alpha(x) v'(x))' + \tilde \gamma (x) u(x) = \tilde f(x).
\]
Comme la fonction $h$ est un polynôme de degrés un elle est bijective de $[-1,1]$ dans $[a,b]$. Alors si $v$ est solution de la dernière EDO on peut inverser la construction en posant $u =h^{-1} \circ v$ on voit que $u$ est solution de la première EDO.



\end{document}



